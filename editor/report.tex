\documentclass[a4paper]{article}
\title{
    Лабораторна робота №1\\
    Взаємодія розподілених процесів через механізм сокетів
}
\author{Віннічук Назар K-25}

\usepackage{graphicx}

\usepackage{polyglossia}
\setdefaultlanguage{ukrainian}
\setotherlanguage{english}

\usepackage{fontspec}
\setmainfont{IBM Plex Sans}
\setmonofont{IBM Plex Mono}

\usepackage{minted}

\usepackage[margin=2cm]{geometry}

\usepackage[parfill]{parskip}

\frenchspacing

\newcommand{\code}[1]{\textbf{\texttt{#1}}}

\begin{document}

\maketitle
\tableofcontents

\section{Умова}

\subsection{Варіант 1: Редактор текстових рядків}

Після приєднання клієнта, сервер передає йому набір текстових рядків (не менше
5). У клієнті можна змінювати символи, вставляти нові чи видаляти, не порушуючи
кількість рядків та їх максимальний розмір. Будь-яка зміна у рядках клієнта
(заміна символу, видалення, вставка нового) передаються серверу для збереження.

\section{Реалізація}

\subsection{Механізм сокетів}
Клієнт та сервер використовують стандартні засоби роботи з сокетами ОС Linux.
Функції та структури для роботи з ними оголошенні в заголовках
\code{<sys/socket.h>} та \code{<netinet/in.h>}. Сервер використовує послідовно
функції \code{socket()}, \code{bind()}, та \code{listen()}, після чого очікує
на клієнтів у циклі з функцією \code{accept()}; при отримані звістки про
від'єднання клієнта викликається \code{close()} та цикл очікування на клієнтів
продовжується. Клієнт натомість викликає лише \code{socket()}, \code{connect()}
та \code{close()}.

\subsection{Редагування тексту}
Після під'єднання клієнта сервер надсилає йому текст до редагування, по рядку
на повідомлення. Потім сервер очікує на команди клієнта. Команда довжини 0 означає
бажання клієнта припинити сеанс редагування. Інші команди мають довжину в 3 байти, які
репрезентують координати та символ. Команди інших довжин вважаються помилковими — сервер
припиняє з'єднання чи ігнорує решту команди у відповідь на закоротку чи задовгу команди
відповідно.

\subsection{Висновки}
\begin{itemize}
    \item
        Виклик функцій \code{write()} та \code{read()} можуть повернути меншу
        кількість байт ніж вимагалось аргументом функції — передача повідомлень
        не гарантовано суцільна. Тому запис та читання повинні реалізовуватись
        у циклі. Прикладом реалізації є макрос \code{do\_till\_end()} у
        \code{shared.h}.
    \item
        Робота з сокетами вимагає виклику кількох функцій у певному порядку. Варто уникати
        типових помилок:
        \begin{itemize}
            \item Відсутність \code{close()} по завершенню з'єднання.
            \item Використання \code{connect()} сервером.
            \item Використання \code{bind()} клієнтом.
            \item Відсутність \code{accept()} сервером по завершенню з'єднання,
                що призводить до неможливості роботи сервером з кількома послідовними
                клієнтами.
        \end{itemize}
    \item Неправильний вибір параметрів-констант (\code{AF\_INET}) та
        неправильне конструювання структури, що описує адресу
        (\code{struct sockaddr\_in addr}), призводить до помилок, що важко ідентифікувати.
\end{itemize}

\section{Програмний код}

\subsection{Makefile}
\inputminted{Makefile}{Makefile}

\subsection{shared.h}
\inputminted{C}{shared.h}

\subsection{client.c}
\inputminted{C}{client.c}

\subsection{server.c}
\inputminted{C}{server.c}

\section{Зображення програм}

\subsection{Сервер}
Сервер використовує інтерфейс командного рядка. Він виводить текст за кожної
його зміни та повідомляє про від'єднання клієнта:

\begin{minted}{text}
$ ./server 2> /dev/null
Eello, this is a text for you to edit!
Here are some words, and more and more...
Blah, blah...
Do not forget to save the changes once you are done!
Just kidding, it is saved automatically:)

Edllo, this is a text for you to edit!
Here are some words, and more and more...
Blah, blah...
Do not forget to save the changes once you are done!
Just kidding, it is saved automatically:)

Edilo, this is a text for you to edit!
Here are some words, and more and more...
Blah, blah...
Do not forget to save the changes once you are done!
Just kidding, it is saved automatically:)

Edito, this is a text for you to edit!
Here are some words, and more and more...
Blah, blah...
Do not forget to save the changes once you are done!
Just kidding, it is saved automatically:)

Editi, this is a text for you to edit!
Here are some words, and more and more...
Blah, blah...
Do not forget to save the changes once you are done!
Just kidding, it is saved automatically:)

Editin this is a text for you to edit!
Here are some words, and more and more...
Blah, blah...
Do not forget to save the changes once you are done!
Just kidding, it is saved automatically:)

Editingthis is a text for you to edit!
Here are some words, and more and more...
Blah, blah...
Do not forget to save the changes once you are done!
Just kidding, it is saved automatically:)

Editing!his is a text for you to edit!
Here are some words, and more and more...
Blah, blah...
Do not forget to save the changes once you are done!
Just kidding, it is saved automatically:)

Editing!!is is a text for you to edit!
Here are some words, and more and more...
Blah, blah...
Do not forget to save the changes once you are done!
Just kidding, it is saved automatically:)

Editing!!!s is a text for you to edit!
Here are some words, and more and more...
Blah, blah...
Do not forget to save the changes once you are done!
Just kidding, it is saved automatically:)

Editing!!!s is a text for you to edit!
Here are some ?ords, and more and more...
Blah, blah...
Do not forget to save the changes once you are done!
Just kidding, it is saved automatically:)

Editing!!!s is a text for you to edit!
Here are some ??rds, and more and more...
Blah, blah...
Do not forget to save the changes once you are done!
Just kidding, it is saved automatically:)

Editing!!!s is a text for you to edit!
Here are some ???ds, and more and more...
Blah, blah...
Do not forget to save the changes once you are done!
Just kidding, it is saved automatically:)

Editing!!!s is a text for you to edit!
Here are some ????s, and more and more...
Blah, blah...
Do not forget to save the changes once you are done!
Just kidding, it is saved automatically:)

Editing!!!s is a text for you to edit!
Here are some ?????, and more and more...
Blah, blah...
Do not forget to save the changes once you are done!
Just kidding, it is saved automatically:)

The client has disconnected
\end{minted}

\subsection{Клієнт}
Клієнт використовує інтерфейс консольної графіки за допомогою бібліотеки ncurses.

\begin{center}
  \includegraphics[scale=0.5]{screenshot.png}
\end{center}

\section{Протокол виконання}

До сервера було під'єднано послідовно два клієнта, що робили зміни у рядках та завершували
з'єднання.

\subsection{Сервер}
\inputminted[fontsize=\footnotesize, breaklines]{text}{server.log}

\subsection{Клієнт №1}
\inputminted[fontsize=\footnotesize, breaklines]{text}{client1.log}

\subsection{Клієнт №2}
\inputminted[fontsize=\footnotesize, breaklines]{text}{client2.log}

\end{document}
